\exercises

\exercise 
Does $\mathbb{Z}_3$, the set of residue classes modulo $3$, form a group with respect to addition? with respect to multiplication?

\answer
Yes, with respect to addition. No, with respect to multiplication. It can be checked exhaustively.

\begin{center}
    \begin{tabular} { c|c c c }
        $+$ & 0 & 1 & 2 \\\hline
        0 & 0 & 1 & 2 \\
        1 & 1 & 2 & 0 \\
        2 & 2 & 0 & 1 \\
    \end{tabular}
    \begin{tabular} { c|c c c }
        $\times$ & 0 & 1 & 2 \\\hline
        0 & 0 & 0 & 0 \\
        1 & 0 & 1 & 2 \\
        2 & 0 & 2 & 1 \\
    \end{tabular}
\end{center}


\exercise
Do the non-zero residue classes modulo 4 form a group with respect to multiplication?

\answer
No it is not a group. $2$ has no inverse modulo $4$.


\exercise
Prove: If $a, b, c \in \mathcal{G}$, then $a \circ b = a \circ c$ (also, $b \circ a = c \circ a$) implies $b = c$

\answer
Multiply on the left by $a^{-1}$. (For the right case, multiply on the right)


\exercise
Prove: When $a, b \in \mathcal{G}$, each of the equations $a \circ x = b$ and $y \circ a = b$ has a unique solution.

\answer
$x = a^{-1} \circ b$ and $y = b \circ a^{-1}$ are solutions. That they are unique follows from the cancellation property; if $x$ and $x'$ are solutions, then $a \circ x = a \circ x'$, hence $x = x'$.


\exercise
Prove: For any $a \in \mathcal{G}$, $a^m \circ a^n = a^{m+n}$ when $m, n \in \mathbb{Z}$

\answer
-


\exercise
Prove: A non-empty subset $\mathcal{G}'$ of a group $\mathcal{G}$ is a subgroup of $G$ if and only if, for all $a,b \in \mathcal{G}'$, $a^{-1} \circ b \in \mathcal{G}'$

\answer
Let $b = 1$. Then the condition shows that $a^{-1} \in \mathcal{G}'$ for all $a \in \mathcal{G}'$. That then implies that $(a^{-1})^{-1} \circ b \in \mathcal{G}'$, so we recover the closure condition $a, b \in \mathcal{G}'$ implies $a \circ b \in \mathcal{G}'$. Hence $\mathcal{G}'$ is a subgroup of $\mathcal{G}$.


\exercise 
Prove: If $S$ is any set of subgroups of a group $\mathcal{G}$, the intersection of these subgroups is also a subgroup of $\mathcal{G}$.

\answer
Let $\mathcal{H} = \bigcap S$. Let $a \in \mathcal{H}$. Then $a^{-1}$ exists in every subgroup in $S$. Hence $a^{-1} \in \mathcal{H}$. Let $a, b \in \mathcal{H}$. Then, again, $a \circ b$ is in every subgroup of $S$. Then $a \circ b \in mathcal{H}$. Then $\mathcal{H}$ is a subgroup of $\mathcal{G}$.


\exercise
Prove: Every subgroup of a cyclic group is itself a cyclic group.

\answer
Let $\mathcal{G}$ be a cyclic group and $\mathcal{H}$ be a subgroup of $\mathcal{G}$. Consider the element $a^m \in \mathcal{H}$ with the least positive $m$. Then take any other element $a^k \in \mathcal{H}$. Then use Euclidean division to find $k = mq + r$. Then
\[a^k = a^{mq+r} = (a^m)^q \circ a^r\]
Then
\[(a^m)^{-q} \circ a^k = a^r\]
Since $a^m \in \mathcal{H}$, $(a^m)^{-q} \in \mathcal{H}$. Then, since $a^k \in \mathcal{H}$ by assumption, $a^r \in \mathcal{H}$. But since $r < m$, $r$ must equal $0$, hence $a^k = (a^m)^q$, and $\mathcal{H}$ is generated by $a^m$


\exercise
The subset $\{\mathbf{u} = (1), \rho, \rho^2, \rho^3, \sigma^2, \tau^2, b = (13), e = (24)\}$ of $S_4$ is a group (see the operation table below), called the \textit{octic group of a square} or the {dihedral group}. We shall now show how this permutation group may be obtained using properties of symmetry of a square.

\answer


\exercise A permutation group on $n$ symbols is called \textit{regular} if each of its elements except the identity moves all $n$ symbols. Find the regular permutation groups on four symbols.

\answer


\exercise
Prove: The mapping $\mathbb{Z} \to \mathbb{Z}_n: m \to [m]$ is a homomorphism of the additive group $\mathbb{Z}$ onto the additive group $\mathbb{Z}_n$ of integers modulo $n$.

\answer


\exercise
In a homomorphism between two groups $\mathcal{G}$ and $\mathcal{G}'$, their identity elements correspond, and if $x \in \mathcal{G}$ and $x' in \mathcal{G}'$ correspond so also do their inverses.

\answer
Let $1$ be the identity of $\mathcal{G}$ and $1'$ the identity of $\mathcal{G}'$. Let $\phi$ be a homomorphism from $\mathcal{G}$ to $\mathcal{G}'$. Let $x$ be any non-identity element of $\mathcal{G}$. Then $1' \square \phi(x) = \phi(x) = \phi(1 \circ x) = \phi(1) \square \phi(x)$. Then $\phi(1) = 1'$.

\exercise
Prove: every cyclic group of infinite order is isomorphic to the additive group $\mathbb{Z}$.

\answer
Let $\mathcal{G}$ be an infinite cyclic group generated by $a$, and consider the homomorphism $\mathbb{Z} \to \mathcal{G}$ defined by $n \to a^n$. This is a homomorphism because $a^{s+t} = a^s a^t$. It is onto, and moreover, is into because it has a trivial kernel. If it didn't, then it means that there existed $m$ such that $a^m = 1$, which would have implied $\mathcal{G}$ is not finite, a contradiction. Hence the map is an isomorphism.


\exercise
Prove: Every finite group of order $n$ is isomorphic to a permutation group on $n$ symbols.

\answer
Consider a group's action on itself by left translation yadda yadda.


\exercise
Prove: The kernel of a homomorphism is a normal subgroup.

\answer
Let $\phi$ be a homomorphism from $G$ to $H$. Let $K$ be the kernel of $\phi$. Let $a, b \in K$. First, we show that $K < G$.

Let $a, b \in K$. Then $\phi(ab) = \phi(a)\phi(b) = 1_H 1_H = 1_H$, so $ab \in K$. Also, $\phi(a^{-1}) = \phi(a)^{-1} = 1_H^{-1} = 1_H$, so $a^{-1} \in K$. 

Hence $K < G$. Next we show that $K$ is normal. Let $g \in G$ and $k \in K$. Then $\phi(g^{-1}kg) = \phi(g)^{-1} 1_H \phi(g) = \phi(g)^{-1} \phi(g) = 1_H$, hence $g^{-1}kg \in K$. Then $K \triangleleft G$


\exercise
Prove: The product of cosets 
\[(Ha)(Hb) = \{(h_1a)(h_2b): h_1,h_2 \in H\}\]
where $H \triangleleft G$ is well defined.

\answer
$(Ha)(Hb) = H(aHb) = (HH)(ab) = H(ab)$


\exercise
Prove: Any quotient group of a cyclic group is cyclic

\answer
Let $G$ be cyclic, and generated by $a$, and let $H \triangleleft G$. Since $G$ is abelian, $(Ha)^m = H^m a^m = Ha^m$. Since every coset of $H$ in $G$ is of the of the form $Ha^m$, we have proven that $Ha$ generates $G/H$.


\exercise
Prove: Every finite group has at least one composition series.

\answer
Lemma: If a finite group has proper normal subgroups, it contains a maximal normal subgroup. Proof: Take the product of proper normal subgroups.

Then just use induction.

\exercise


\answer


