\section{Groups}

\definition
A non-empty set $\mathcal{G}$ equipped with a binary operation $\circ$ is a \textit{group} if 
\begin{itemize}
    \item[$\mathbf{P}_1$:] $(a \circ b) \circ c = a \circ (b \circ c)$ (Associativity)
    \item[$\mathbf{P}_2$:] There exists an element $1 \in \mathcal{G}$ such that $1 \circ a = a \circ 1 = a$ for all $a \in \mathcal{G}$ (Unit)
    \item[$\mathbf{P}_3$:] For all $a \in \mathcal{G}$, there exists an element $a^{-1}$ such that $a \circ a^{-1} = a^{-1} \circ a = 1$ (Inverse)
\end{itemize}

\example
\begin{itemize}
    \item[(a)] The set $\mathbb{Z}$ of all integers. $+$ is the operation, $0$ is the identity element and the inverse of $a$ is $-a$. This is the \textit{additive group} $\mathbb{Z}$.
    \item[(b)] 
\end{itemize}


\section{Simple Properties of Groups}

\theorem (Left Cancellation) \\
Let $a,b,c \in \mathcal{G}$. Then $a \circ b = a \circ c$ implies $b = c$.

\theorem (Latin Square Property) \\ 
Let $a, b \in \mathcal{G}$, then the equations 
\[ax = b\]
\[ya = b\]
have unique solutions $x$ and $y$ respectively.

\theorem (Involution) \\
For all $a \in \mathcal{G}$, $(a^{-1})^{-1} = a$.

\theorem  \\
Let $a, b \in \mathcal{G}$. Then $(a \circ b)^{-1} = b^{-1} \circ a^{-1}$

\theorem (Inverse of composition is reversed composition of inverse) \\
Let $a, b, \ldots, p, q \in \mathcal{G}$, then 
\[(a \circ b \circ \cdots \circ p \circ q)^{-1} = p^{-1} \circ q^{-1} \circ \cdots \circ b^{-1} \circ a^{-1}\]

\theorem 
For all $a \in \mathcal{G}$ and $m, n \in \mathbb{Z}$, 
\[a^m \circ a^n = a^{m+n}\]
\[(a^m)^n = a^{mn}\]

\definition The \textit{order of a group} $\mathcal{G}$ is the number of elements in $\mathcal{G}$.

\definition The \textit{order of an element} $a \in \mathcal{G}$ is the least positive integer $n$ such that $a^n = 1$.

\definition If $a \in \mathbb{Z}$ is not $0$, then the order of $a$ is infinite.


\section{Subgroups}

\definition 
Let $\mathcal{G}$ be a group with the operation $\circ$. A subset $\mathcal{H}$ of $\mathcal{G}$ is a \textit{subgroup of} $\mathcal{G}$ if $\mathcal{H}$ is also a group with the operation $\circ$ (restricted to $\mathcal{H}$).

Every group $\mathcal{G}$ has two trivial subgroups: $\{1\}$ and $G$.

\theorem 
$\mathcal{G}'$ is a subgroup of $\mathcal{G}$ if and only if
\begin{itemize}
    \item[(i)] 
        $a, b \in \mathcal{G}'$ implies $a \circ b \in \mathcal{G}'$
    \item[(ii)]
        $a \in \mathcal{G}'$ implies $a^{-1} \in \mathcal{G}'$
\end{itemize}

\theorem
$\mathcal{G}'$ is a subgroup of $\mathcal{G}$ if and only if $a, b \in \mathcal{G}'$ implies $a^{-1} \circ b \in \mathcal{G}'$

\theorem
$\{a^n : n \in \mathbb{Z}\}$ is a subgroup of $\mathcal{G}$ for all $a \in \mathcal{G}$

\theorem
The intersection of any set of subgroups of $\mathcal{G}$ is a subgroup of $\mathcal{G}$.


\section{Cyclic groups}

\definition 
A group is \textit{cyclic} if it is generated by a single element $a$.

\theorem
If $\mathcal{G}$ is a cyclic group of order $n$ generated by $a$ then $a^t$ is a generator of $\mathcal{G}$ if and only if $\gcd(n, t) = 1$

\theorem
Every subgroup of a cyclic group is cyclic.


\section{Permutation groups}

$S_n$ is the symmetric group.


\section{Homomorphisms}

\definition
If $\mathcal{G}$ is a group with operation $\circ$ and $\mathcal{H}$ is a group with operation $\square$, a homomorphism between $\mathcal{G}$ and $\mathcal{H}$ is a function $\phi: \mathcal{G} \to \mathcal{H}$ such that for all $a, b \in \mathcal{G}$,
\[\phi(a \circ b) = \phi(a) \square \phi(b)\]

\theorem
Letting $\mathcal{G}$, $\mathcal{H}$, and $\phi$ be defined as above, $\phi(1_\mathcal{G}) = 1_\mathcal{H}$ and $\phi(a^{-1}) = \phi(a)^{-1}$ for all $a \in \mathcal{G}$.

\theorem
The homomorphic image of a cyclic group is cyclic


\section{Isomorphisms}

\definition
If $\phi$ happens to also be a bijection of sets, then $\phi$ is called an \textit{isomorphism} and $\mathcal{G}$ and $\mathcal{H}$ are said to be \textit{isomorphic}

\theorem
\begin{itemize}
    \item[(a)]
        Every cyclic group of infinite order is isomorphic to $\mathbb{Z}$
    \item[(b)]
        Every cyclic group of order n is isomorphic to $\mathbb{Z}_n$
\end{itemize}

\theorem (Cayley's Theorem)
Every finite group of order $n$ is isomorphic to a subgroup of $S_n$


\section{Cosets}

\definition
Let $\mathcal{G}$ be a finite group with operation $\circ$, $H$ be a subgroup of $\mathcal{G}$,and $a \in \mathcal{G}$. The \textit{right coset of $H$ generated by $a$} $Ha$ is
\[Ha := \{h \circ a : h \in H\}\]
Similarly, the \textit{left coset of $H$ generated by $a$} is
\[aH := \{a \circ h : h \in H\}\]


