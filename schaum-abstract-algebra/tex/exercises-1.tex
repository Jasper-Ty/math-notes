\exercises

\exercise 
Exhibit in tabular form: 
\begin{itemize}
    \item[(a)] 
        $A = \{a : a \in \mathbb{N}, 2 < a < 6\}$
    \item[(b)] 
        $B = \{p : p \in \mathbb{N}, p < 10, p \text{ is odd}\}$
    \item[(c)] 
        $C = \{x : x \in \mathbb{Z}, 2x^2 + x - 6 = 0\}$
\end{itemize}

\answer
\begin{itemize}
    \item[(a)] 
        $A = \{3, 4, 5\}$
    \item[(b)] 
        $B = \{9, 7, 5, 3, 1\}$
    \item[(c)] 
        $C = \{-2\}$
\end{itemize}


\exercise 
Let $A = \{a,b,c,d\}$, $B = \{a,c,g\}$, $C = \{c,g,m,n,p\}$. Then $A \cup B = \{a,b,c,d,g\}$, $A \cup C = \{a,b,c,d,g,m,n,p\}$, $B \cup C = \{a,c,g,m,n,p\}$;

\answer 
\large\textbf{WHAT IS THE QUESTION?}\normalsize


\exercise 
Consider the subsets $K = \{2,4,6,8\}$, $L=\{1,2,3,4\}$, $M=\{3,4,5,6,8\}$ of $U = \{1,2,3,\ldots,10\}$.
\begin{itemize}
    \item[(a)]
        Exhibit $K'$, $L'$, $M'$ in tabular form.
    \item[(b)]
        Show that $(K \cup L)' = K' \cap L'$
\end{itemize}

\answer
\begin{itemize}
    \item[(a)]
        $K' = \{1,3,5,7,9,10\}$ \\
        $L' = \{5,6,7,8,9,10\}$ \\
        $U' = \{1,2,7,9,10\}$
    \item[(b)]
        $K \cup L = \{1,2,3,4,6,8\}$, so $(K \cup L)' = \{5,7,9,10\}$. Using the above, $K' \cap L' = \{5,7,9,10\}$.
\end{itemize}


\exercise Skip \\
\exercise Skip \\
\exercise Skip \\
\exercise Skip


\exercise 
Prove $(A \cup B) \cup C = A \cup (B \cup C)$

\answer 
Any element $x$ belongs to the left hand side if $(x \in A \vee x \in B) \vee x \in C$. It belongs to the right hand side if $x \in A \vee (x \in B \vee x \in C)$. Both expressions are logically equivalent. Then the left set is equal to the right set by the axiom of extensionality.


\exercise
Prove $(A \cap B) \cap C = A \cap (B \cap C)$

\answer 
Similar answer as the previous-- follows from the associativity of $\wedge$


\exercise
Prove $A \cap (B \cup C) = (A \cap B) \cup (A \cap C)$

\answer
Follows from $\wedge$ distributing over $\vee$


\exercise
Prove $(A \cap B)' = A' \cap B'$

\answer
Follows from DeMorgan's laws


\exercise
Prove $A \cup (B \cap C) = (A \cup B) \cap (A \cup C)$

\answer
Follows from $\vee$ distributing over $\wedge$


\exercise
Prove $A - (B \cup C) = (A - B) \cap (A - C)$

\answer
Follows again from DeMorgan's laws (for classical logic): $x$ belongs to the left hand side if
\[x \in A \wedge \neg(x \in B \vee x \in C)\] 
Which is equivalent to 
\[x \in A \wedge (x \notin B \wedge x \notin C)\] 
Then, since $\wedge$ distributes over itself, we can rewrite this as
\[(x \in A \wedge x \notin B) \wedge (x \in A \wedge x \notin C)\] 
Which is the right hand side.


\exercise
Prove: $(A \cup B) \cap B' = A$ if and only if $A \cap B = \varnothing$

\answer
By distributivity,
\[B' \cap (A \cup B) = (B' \cap A) \cup (B' \cap B)\]
Then,
\[(B' \cap A) \cup (B' \cap B) = (B' \cup A) \cap \varnothing = B' \cap A\ = A - B\]
Hence the equation is equivalent to 
\[A- B = A\]
Which is only true if and only if $A \cap B = \varnothing$


\exercise
Prove that $X \subseteq Y$ if and only if $Y' \subseteq X$

\answer
$X \subseteq Y$ when $a \in X \implies a \in Y$. By contraposition, this is equivalent to $a \notin Y \implies a \notin X$, which is the definition of $Y' \subseteq X'$


\exercise
Prove the identity $(A - B) \cup (B- A) = (A \cup B) - (A \cap B)$ of Example 10 using the identity $A - B = A \cap B'$ of Example 9

\answer
\begin{align*}
    (A \cup B) - (A \cap B) &= (A \cup B) \cap (A \cap B)' \\
                            &= (A \cup B) \cap (A' \cup B') \\
                            &= ((A \cup B) \cap A') \cup ((A \cup B) \cap B') \\
                            &= ((A \cap A') \cup (B \cap A')) \cup ((A \cap B') \cup (B \cap B')) \\
                            &= (\varnothing \cup (B-A)) \cup ((A-B) \cup \varnothing) \\
                            &= (A-B) \cup (B-A)
\end{align*}


\exercise
In Fig. 1-8, show that any two line segments have the same number of points.

\answer
Book has a geometric solution. 


\exercise
Prove: 
\begin{itemize}
    \item[(a)]
        $x \to x+2$ is a mapping of $\mathbb{N}$ into, but not onto, $\mathbb{N}$.
    \item[(b)]
        $x \to 3x - 2$ is a one-to-one mapping of $\mathbb{Q}$ onto $\mathbb{Q}$
    \item[(c)]
        $x \to x^3 - x^2 - x$ is a mapping of $\mathbb{R}$ onto $\mathbb{R}$ but not one-to-one.
\end{itemize}

\answer
\begin{itemize}
    \item[(a)]
        By the cancellation law in $\mathbb{N}$, $x + 2 = y + 2$ implies $x = y$, so the map is injective. But there is no natural number $x$ such that $x+2 = 1$. Hence the map cannot be onto.
    \item[(b)]
        By the cancellation laws again for $\mathbb{Q}$, $3x - 2 = 3y - 2$ imply $x = y$, so this map is into. Moreover, for all $h \in \mathbb{Q}$, the map sends the number $(h+2)/3$ to $h$, hence the map is surjective.
    \item[(c)]
        The map is equivalent to 
        \[x \to (x)(x - \phi^+)(x - \phi^-)\]
        where $\phi^+$ and $\phi^-$ are the roots to the equation $x^2 - x - 1 = 0$. Hence the map is not injective, as multiple $x$ map to $0$.
        The map is surjective, because the equation $x^3 - x^2 - x - r = 0$ always has a real root.
\end{itemize}


\exercise
Prove: If $\alpha$ is a one-to-one mapping of a set $S$ onto a set $T$, then $\alpha$ has a unique inverse and conversely.

\answer
Suppose $\alpha$ is one-to-one. Let $t \in T$. There must exist $s \in S$ such that $\alpha(s) = t$, since $\alpha$ is onto. Suppose another such element $s' \in S$ existed, then $\alpha(s') = t = \alpha(s)$, so it must be that $s = s'$, hence $s$ is unique. Define $a^{-1}(t) = s$ for all $t \in T$. Then $\alpha \circ \alpha^{-1} = \text{id}$. 

$\alpha$ must be unique, since inverse elements in any binary operation are unique.


\exercise 
Prove: If $\alpha$ is a one-to-one mapping of a set $S$ onto a set $T$ and $\beta$ is a one-to-one mapping of $T$ onto a set $U$, then $(\alpha\beta)^{-1} = \beta^{-1} \alpha^{-1}$

\answer
$\alpha \beta \beta^{-1} \alpha^{-1} = \text{id}$


\subsection*{Supplementary Problems}

\exercise
Exhibit each of the following in tabular form:
\begin{enumerate}
    \item[(a)]
        the set of negative integers greater than $-6$
    \item[(b)]
        the set of integers between $-3$ and $4$,
    \item[(c)]
        the set of integers whose squares are less than $20$,
    \item[(d)]
        the set of all positive factors of $18$,
    \item[(e)]
        the set of all common factors of $16$ and $24$,
    \item[(f)]
        $\{p:p \in \mathbb{N}, p^2 < 10\}$
    \item[(g)]
        $\{b: b \in \mathbb{N}, 3 \leq b \leq 8\}$
    \item[(h)]
        $\{x: x \in \mathbb{Z}, 3x^2 + 7x + 2 = 0\}$
    \item[(i)]
        $\{x: x \in \mathbb{Q}, 2x^2 + 5x + 3 = 0\}$
\end{enumerate}

\answer
\begin{enumerate}
    \item[(a)]
        $\{-5, -4, -3, -2, -1\}$
    \item[(b)]
        $\{-2, -1, 0, 1, 2, 3\}$
    \item[(c)]
        $\{-4, -3, -2, -1, 0, 1, 2, 3, 4\}$
    \item[(d)]
        $\{1, 2, 3, 6, 9, 18\}$
    \item[(e)]
        $\{1, 2, 4, 8\}$
    \item[(f)]
        $\{1, 2, 3\}$
    \item[(g)]
        $\{3, 4, 5, 6, 7, 8\}$
    \item[(h)]
        $3x^2 + 7x + 2 = (3x+1)(x+2)$, so the roots are $-2$ and $-1/3$. So our set in tabular form is $\{-2\}$
    \item[(i)]
        $2x^2 + 5x + 3 = (2x + 3)(x + 1)$, so the roots are $-3/2$ and $-1$. So our set in tabular form is $\{-3/2, -1\}$
\end{enumerate}


\exercise
Verify: 
\begin{itemize}
    \item[(a)]
        $\{x : x \in \mathbb{N}, x < 1\} = \varnothing$
    \item[(a)]
        $\{x : x \in \mathbb{Z}, 6x^2 + 5x - 4\} = \varnothing$
\end{itemize}

\answer
\begin{itemize}
    \item[(a)]
        $x \geq 1$ for all $x \in \mathbb{N}$. 
    \item[(a)]
        $6x^2 + 5x - 4 = (3x + 4)(2x - 1)$. So the roots are $1/2$ and $-4/3$, which are non-integers.
\end{itemize}


\exercise
Exhibit the 15 proper subsets of $S = \{a,b,c,d\}$

\answer
$\varnothing$, $\{a\}$,  $\{b\}$,  $\{c\}$,  $\{d\}$,  $\{a, b\}$,  $\{a, c\}$,  $\{a, d\}$,  $\{b, c\}$,  $\{b, d\}$,  $\{c, d\}$, $\{a, b, c\}$, $\{a, b, d\}$, $\{a, c, d\}$, $\{b, c, d\}$ 


\exercise
Show that the number of proper subsets of $S = \{a_1, a_2, \ldots, a_n\}$ is $2^n-1$. 

\answer
The number of subsets of $S$ is counted by the number of functions from $S$ to $\{0, 1\}$, which is the set $2^S$, whose cardinality is $2^{|S|} = 2^n$. Minus the set $S$ itself, we have $2^n - 1$.


\exercise
Using the sets of Problem 1.2, verify
\begin{itemize}
    \item[(a)]
        $(A \cup B) \cup C = A \cup (B \cup C)$
    \item[(b)]
        $(A \cap B) \cap C = A \cap (B \cap C)$
    \item[(c)]
        $(A \cup B) \cap C \neq A \cup (B \cap C)$
\end{itemize}

\answer
\begin{itemize}
    \item[(a)]
        \begin{align*}
            (A \cup B) \cup C &= \{a, b, c, d, g\} \cup \{c, g, m, n, p\} \\
                              &= \{a, b, c, g, m, n, p\} \\
                              &= \{a, b, c, d\} \cup \{a, c, g, m, n, p\} \\
                              &= A \cup (B \cup C)
        \end{align*}
    \item[(b)]
        $(A \cap B) \cap C = A \cap (B \cap C)$
    \item[(c)]
        $(A \cup B) \cap C \neq A \cup (B \cap C)$
\end{itemize}


\exercise

\answer
