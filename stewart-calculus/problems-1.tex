\documentclass{article}

\usepackage{mystyle}

\begin{document}

\begin{mytitle}
    Calculus \\
    Problems 1.1 \\
    \normalsize Jasper Ty
\end{mytitle}

\begin{Answer}
    Let $a$ be the side of length $4$. Let the adjacent side be $b$ and the hypotenuse $c$.
    The altitude perpendicular to the hypotenuse splits the triangle into two similar triangles (proof via the angle sum) that are similar to the original triangle. $s$ is the hypotenuse of one of them, $a$ is the hypotenuse of the other. Hence, if the length of the altitude is $x$, we have the relationship
    \[x / a = b / c\]
    Simplifiying
    \[x = \frac{4\sqrt{c^2-16}}{c}\]
\end{Answer}

\begin{Answer}
    The perimeter of a triangle is 
    \[P = a + b + c\]
    Then
    \[P^2 = a^2 + b^2 + c^2 + 2ab + 2ac + 2bc\]
    Replacing $a^2 + b^2 = c^2$,
    \[P^2 = c^2 + c^2 + 2ab + 2ac + 2bc = 2c^2 + 2ab + 2ac + 2bc\]
    Collecting a common $c$,
    \[P^2 = 2c(a + b + c) + 2ab\]
    Replacing $a + b + c = P$
    \[P^2 = 2Pc + 2ab\]
    Using the previously found identity for the altitude perpendicular to the hypotenuse $A = ab/c$,
    \[P^2 = 2Pc + 2(12c)\]
    Then 
    \[P^2 = 2c \cdot P + 2c \cdot 12\]
    \[P^2 = 2c(P + 12)\]
    Solving for $c$,
    \[\frac{P^2}{2(P+12)} = c\]
\end{Answer}

\begin{Answer}
    We find the different regions of interest:

\end{Answer}


\end{document}
