\documentclass{article}

\usepackage{mystyle}

\begin{document}

\begin{mytitle}
    Calculus \\
    Problems 1.1 \\
    \normalsize Jasper Ty
\end{mytitle}

\begin{Answer}
    Let $a$ be the side of length $4$. Let the adjacent side be $b$ and the hypotenuse $c$.
    The altitude perpendicular to the hypotenuse splits the triangle into two similar triangles (proof via the angle sum) that are similar to the original triangle. $s$ is the hypotenuse of one of them, $a$ is the hypotenuse of the other. Hence, if the length of the altitude is $x$, we have the relationship
    \[x / a = b / c\]
    Simplifiying
    \[x = \frac{4\sqrt{c^2-16}}{c}\]
\end{Answer}

\begin{Answer}
    The perimeter of a triangle is 
    \[P = a + b + c\]
    Then
    \[P^2 = a^2 + b^2 + c^2 + 2ab + 2ac + 2bc\]
    Replacing $a^2 + b^2 = c^2$,
    \[P^2 = c^2 + c^2 + 2ab + 2ac + 2bc = 2c^2 + 2ab + 2ac + 2bc\]
    Collecting a common $c$,
    \[P^2 = 2c(a + b + c) + 2ab\]
    Replacing $a + b + c = P$
    \[P^2 = 2Pc + 2ab\]
    Using the previously found identity for the altitude perpendicular to the hypotenuse $A = ab/c$,
    \[P^2 = 2Pc + 2(12c)\]
    Then 
    \[P^2 = 2c \cdot P + 2c \cdot 12\]
    \[P^2 = 2c(P + 12)\]
    Solving for $c$,
    \[\frac{P^2}{2(P+12)} = c\]
\end{Answer}

\begin{Answer}
    This gives us four possible equations, one of which is impossible and omitted.
    \[(2x - 1) - (x + 5) = 3, \qquad \frac{1}{2} < x\] 
    \[-(2x - 1) - (x + 5) = 3, \qquad -5 < x \leq \frac{1}{2}\]
    \[-(2x + 1) + (x + 5) = 3, \qquad x \leq -5\]
    Solving,
    \[x = 9\]
    \[x = -\frac{7}{3}\]
    \[x = 3\]
    The third one is a contradiction, so the solutions are $x = 9$ and $x = -7/3$
\end{Answer}

\begin{Answer}
    By the reverse triangle inequality,
    \[2 = |x - x + 2| = |(x - 1) - (x - 3)| \geq \bigg| |x-1| - |x-3|\bigg|\]

    Hence $|x-1| - |x-3| \leq 2$, which implies that there are no solutions to $|x-1| - |x-3| \geq 5$
\end{Answer}


\begin{Answer}
    Skip
\end{Answer}

\begin{Answer}
    Skip
\end{Answer}

\begin{Answer}
    Skip
\end{Answer}

\begin{Answer}
    Skip
\end{Answer}

\begin{Answer}
    Skip
\end{Answer}

\begin{Answer}
    Skip
\end{Answer}

\begin{Answer}
    \[(\log_23)(\log_34)\cdots(\log_{31}32)\]
    is, by change of base, equal to
    \[\frac{\ln3}{\ln2}\frac{\ln4}{\ln3}\cdots\frac{\ln32}{\ln31}\]
    Which is, after cancellations,
    \[\frac{\ln32}{\ln2} = \log_232 = 5\]
\end{Answer}

\begin{Answer}
    \begin{itemize}
        \item[(a)]
            \begin{align*}
                f(-x) &= \ln(-x + \sqrt{(-x)^2 + 1}) \\
                      &= \ln(-x + \sqrt{x^2 + 1}) \\
                      &= \ln((x + \sqrt{x^2 + 1})^{-1}) \\
                      &= -\ln(x + \sqrt{x^2 + 1}) \\
                      &= -f(x)
            \end{align*}
        \item[(b)]
            \begin{align*}
                y &= \ln(x + \sqrt{x^2+1}) \\
                e^y &= x + \sqrt{x^2+1} \\
                e^y - e^{-y} &= (x + \sqrt{x^2+1}) - (-x + \sqrt{x^2+1}) \\
                e^y - e^{-y} &= 2x \\
                \frac{e^y - e^{-y}}{2} &= x \\
            \end{align*}
    \end{itemize}
\end{Answer}

\begin{Answer}
    By the monotonicity of $\ln$, and the fact that $\ln1 = 0$,
    \[\ln(x^2 - 2x - 2) \leq 0\]
    \[\ln(x^2 - 2x - 2) \leq \ln 1\]
    \[x^2 - 2x - 2 \leq 1\]
    \[x^2 - 2x - 3 \leq 0\]
    \[(x-3)(x+1) \leq 0\]
    Hence it must be that $-1 \leq x \leq 3$
\end{Answer}

\begin{Answer}
    Suppose $\log_25$ is rational. Then $\log_25 = p/q$ for $p, q \in \mathbb{Z}$ and $p, q$ coprime. Then
    \[2^{p/q} = 5\]
    Then,
    \[2^p = 5^q\]
    Which is impossible, as neither side share any prime factors.
\end{Answer}

\begin{Answer}
    Let the distance travelled be $D$. Then the time taken in the first half $T_1$ and the second half $T_2$ is 
    \[T_1 = \frac{D/2}{30} \qquad T_2 = \frac{D/2}{60}\]
    Hence $T_1/T_2 = 60/30 = 2$. Then the average speed is
    \[(30\cdot 2 + 60 \cdot 1)/3 = 40\]
\end{Answer}

\begin{Answer}
    No. Let $f$ be a function such that $f(1) := 0$, $f(2) = 1$. Then let $g$ and $h$ be the constant functions $g := 1$ and $h := 1$. Then $(f \circ (g+h))(x) = f(2) = 1$, but $(f \circ g + f \circ h)(x) = f(1) + f(1) = 0$
\end{Answer}

\begin{Answer}
    \[7 \equiv 1 \mod 6\]
    \[7^n \equiv 1 \mod 6\]
    \[7^n - 1 \equiv 0 \mod 6\]
\end{Answer}

\begin{Answer}
    \[(n+1)^2 - n^2 = (n+1 - n)(n+1 +n) = 2n+1 = 2(n+1) - 1\]
    Hence if $n^2 = 1 + 3 + 5 + \cdots + (2n-1)$, 
    \[1 + 3 + 4 + \cdots + (2n-1) + (2(n+1) - 1) = n^2 + (2(n+1) -1) = (n+1)^2\]
    Since the identity is easily verified for $n=1$, it follows by induction that it is true for all $n$.
\end{Answer}

\begin{Answer}
    \[f_{n+1}(x) = f_0(f_n(x)) = [f_n(x)]^2 = x^{2(n+2)}\]
    Reindexing, $f_n(x) = x^{2n+2}$
\end{Answer}

\begin{Answer}

\end{Answer}
\end{document}
