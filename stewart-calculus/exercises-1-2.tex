\documentclass{article}

\usepackage{mathrsfs}
\usepackage{amsmath}
\usepackage{amssymb}
\usepackage[margin=0.5in]{geometry}

\newenvironment{myboxed}{\noindent\begin{tabular}{|p{.975\linewidth}|}\hline \\}{\\\\\hline\end{tabular}}
\newenvironment{mytitle}{\noindent\large\begin{flushright}}{\end{flushright}\normalsize}

\def\theChapter{1.2}

\newcounter{Answer}
\newenvironment{Answer} {\refstepcounter{Answer}\par\noindent\textbf{A\theChapter.\theAnswer:}} {\bigskip}

\begin{document}

\begin{mytitle}
    Calculus \\
    Chapter 1.1 \\
    \normalsize Jasper Ty
\end{mytitle}

\begin{Answer}
    \begin{itemize}
        \item[(a)] 
            Power function, $a = 1/5$
        \item[(b)]
            Algebraic function
        \item[(c)]
            Polynomial, degree $9$
        \item[(d)]
            Rational function
        \item[(e)]
            Trigonometric function
        \item[(f)]
            Logarithmic function
    \end{itemize}
\end{Answer}

\begin{Answer}
    \begin{itemize}
        \item[(a)] 
            Rational function
        \item[(b)]
            Algebraic function
        \item[(c)]
            Exponential function, $a = 10$
        \item[(d)]
            Power function (polynomial)
        \item[(e)]
            Polynomial
        \item[(f)]
            Trigonometric
    \end{itemize}
\end{Answer}

\begin{Answer}
    \begin{itemize}
        \item[(a)] $h$
        \item[(b)] $f$
        \item[(c)] $g$
    \end{itemize}
\end{Answer}

\begin{Answer}
    \begin{itemize}
        \item[(a)]
            $G$
        \item[(b)]
            $f$
        \item[(c)]
            $F$
        \item[(d)]
            $g$
    \end{itemize}
\end{Answer}

\begin{Answer}
    \begin{itemize}
        \item[(a)]
            $y = 2x + b$
        \item[(b)]
            $y - 1 = m(x - 2)$
        \item[(c)]
            $y-1 = 2(x-2)$
    \end{itemize}
\end{Answer}

\begin{Answer}
    They all cross the point $(-3, 1)$
\end{Answer}

\begin{Answer}
    They all have slope $-1$
\end{Answer}

\begin{Answer}
    \begin{itemize}
        \item[(a)]
        \item[(b)]
            The slope represents how many fewer spaces he can rentfor each dollar he charges. The $y$ intercept of the graph is the maximum number of spaces he can rent. The $x$ intercept represents the minimum amount of rent needed to result in no rentable spaces.
    \end{itemize}
\end{Answer}

\begin{Answer}
    \begin{itemize}
        \item[(a)]
        \item[(b)]
            The slope, $9/5$, is the ratio between the magnitude of a unit of Fahrenheit and a unit of Celsius. The $F$-intercept is $0$ degrees celsius in Fahrenheit.
    \end{itemize}
\end{Answer}

\begin{Answer}
    \begin{itemize}
        \item[(a)]
            $d = 4t/5$, where $t$ is minutes past 2:00
        \item[(b)]
        \item[(c)]
            The slope is Jason's speed.
    \end{itemize}
\end{Answer}

\begin{Answer}
    \begin{itemize}
        \item[(a)]
            \[T - 70 = \frac{80-70}{173-113}(N - 113)\]
            \[T = \frac{1}{6}(N - 113) + 70\]
        \item[(b)]
            The slope of the graph is the reciprocal of how much chirping increases when you increase temperature.
        \item[(c)]
        \[T = \frac{1}{6}(150 - 113) + 70 = \frac{37}{6} + 70 = 76 + \frac{1}{6} \]
    \end{itemize}
\end{Answer}

\begin{Answer}
    \begin{itemize}
        \item[(a)]
            \[C - 2200 = \frac{4800-2200}{300-100}(N-100)\]
            \[C = 13(N-100) + 2200\]
        \item[(b)]
            The slope of the graph is how much more the manufacturing cost increases by adding producing one more chair.
        \item[(c)]
            The $y$-intercept represents the ineliminable overhead with producing chairs.
    \end{itemize}
\end{Answer}

\begin{Answer}
    \begin{itemize}
        \item[(a)]
            \[P = 15 + \frac{4.34}{10}d = 15 + 0.434d\]
        \item[(b)]
            \[100 = 15 + 0.434d\]
            \[d = \frac{100-15}{0.434} \approx 195.85\]
    \end{itemize}
\end{Answer}

\begin{Answer}
    \begin{itemize}
        \item[(a)]
            \[C - 380 = \frac{460-380}{800-480}(d-480)\]
            \[C = \frac{d}{4} + 260\]
        \item[(b)]
            \[C = \frac{1500}{4} + 260 = 635\]
        \item[(c)]
        \item[(d)]
        \item[(e)]
    \end{itemize}
\end{Answer}

\begin{Answer}
    \begin{itemize}
        \item[(a)]
            Trig function
        \item[(b)]
            Linear function
    \end{itemize}
\end{Answer}

\begin{Answer}
    \begin{itemize}
        \item[(a)]
            Power function or a polynomial
        \item[(b)]
            Power function
    \end{itemize}
\end{Answer}

\begin{Answer}
    Skip
\end{Answer}

\begin{Answer}
    Skip
\end{Answer}

\begin{Answer}
    Skip
\end{Answer}

\begin{Answer}
    Skip
\end{Answer}

\begin{Answer}
    Skip
\end{Answer}

\begin{Answer}
    Skip
\end{Answer}
\end{document}
