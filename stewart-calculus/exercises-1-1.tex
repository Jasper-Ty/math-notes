\documentclass{article}

\usepackage{mathrsfs}
\usepackage{amsmath}
\usepackage{amssymb}
\usepackage[margin=0.5in]{geometry}

\newenvironment{myboxed}{\noindent\begin{tabular}{|p{.975\linewidth}|}\hline \\}{\\\\\hline\end{tabular}}
\newenvironment{mytitle}{\noindent\large\begin{flushright}}{\end{flushright}\normalsize}

\def\theChapter{1}

\newcounter{Answer}
\newenvironment{Answer} {\refstepcounter{Answer}\par\noindent\textbf{A\theChapter.\theAnswer:}} {\bigskip}

\begin{document}

\begin{mytitle}
    Calculus \\
    Chapter 1.1 \\
    \normalsize Jasper Ty
\end{mytitle}
\begin{Answer}
    \begin{itemize}
        \item[(a)]
            $f(-1) = -2$
        \item[(b)]
            $f(2) \approx 1.8$
        \item[(c)]
            $f(x) = 2$ at $x = 1$ and $x = -3$
        \item[(d)]
            $f(x) = 0$ at $x \approx -2.5$ and $x \approx 0.4$
        \item[(e)]
            The domain of $f$ is $[-3, 3]$ and the range of $f$ is $[-2, 3]$
        \item[(f)]
            $f$ is increasing on $[-1, 3]$
    \end{itemize}
\end{Answer}

\begin{Answer}
    \begin{itemize}
        \item[(a)] $f(-4) = -2$ and $g(3) = 4$
        \item[(b)]
            $f(x) = g(x)$ at $x = -2$ and $x = 2$
        \item[(c)]
            $f(x) = -1$ at $x = -3$ and $x = 4$
        \item[(d)]
            $f$ is decreasing on $[0, 4]$
        \item[(e)]
            The domain of $f$ is $[-4, 4]$ and the range of $f$ is $[-2, 3]$
        \item[(f)]
            The domain of $g$ is $[-4, 3]$ and the range of $g$ is $[0.5, 4]$
    \end{itemize}
\end{Answer}

\begin{Answer}
    The range of the vertical ground acceleration function is about $[-70, 120]$ 

    The range of the north-south ground acceleration function is about $[-300, 420]$ 

    The range of the east-west ground acceleration function is about $[-200, 200]$ 
\end{Answer}

\begin{Answer}
    Skip
\end{Answer}

\begin{Answer}
    Is not a function. Fails the vertical line test. 
\end{Answer}

\begin{Answer}
    Is a function. The domain is $[-2, 2]$ and the range is $[-1, 2]$
\end{Answer}

\begin{Answer}
    Is a function. The domain is $[-3, 2]$ and the range is $[-3, -2) \cup [-1, 3]$
\end{Answer}

\begin{Answer}
    Is not a function. Fails the vertical line test.
\end{Answer}

\begin{Answer}
    Skip
\end{Answer}

\begin{Answer}
    Skip
\end{Answer}

\begin{Answer}
    Skip
\end{Answer}

\begin{Answer}
    Skip
\end{Answer}

\begin{Answer}
    Skip
\end{Answer}

\begin{Answer}
    Skip
\end{Answer}

\begin{Answer}
    Skip
\end{Answer}

\begin{Answer}
    Skip
\end{Answer}

\begin{Answer}
    Skip
\end{Answer}

\begin{Answer}
    Skip
\end{Answer}

\begin{Answer}
    \begin{itemize}
        \item $f(2) = 3(2)^2 - 2 + 2 = 3\cdot 4 = 12$
        \item $f(-2) = 3(-2)^2 - (-2) + 2 = 3\cdot 4 + 4 = 16$
        \item $f(a) = 3a^2 - a + 2$
        \item $f(-a) = 3(-a)^2 - (-a) + 2 = 3a^2 + a + 2$
        \item $f(a+1) = 3(a+1)^2 - (a+1) + 2 = 3(a^2 + 2a + 1) - a + 3 = 3a^2 + 5a + 6$
        \item $2f(a) = 2(3a^2 - a + 2) = 6a^2 - 2a + 4$
        \item $f(2a) = 3(2a)^2 - 2a + 2 = 3\cdot 4a^2 - 2a + 2 = 12a^2 - 2a + 2$
        \item $f(a^2) = 3(a^2)^2 - 2a^2 + 2 = 3a^4 - 2a^2 + 2$
        \item $[f(a)]^2 = [3a^2 - a + 2]^2 = 9a^4 + a^2 + 4 - 3a^3 + 6a^2 - 2a = 9a^4 + 7a^2 - 3a^3 - 2a + 4$
        \item $f(a+h) = 3(a+h)^2 - (a+h) + 2 = 3(a^2 + 2ah + h^2) - a - h + 2 = 3a^2 - 2ah + h^2 - a - h + 2$
    \end{itemize}
\end{Answer}

\begin{Answer}
    Let $f(r) := V(r+1) - V(r)$ 
    \begin{align*}
        f(r) &= \frac{4}{3} \pi (r+1)^3 - \frac{4}{3} \pi r^3 \\
             &= \frac{4}{3}\pi \bigg( (r+1)^3 - r^3\bigg) \\
             &= \frac{4}{3}\pi \bigg( r^3 + 3r^2 + 3r + 1 - r^3\bigg) \\
             &= \frac{4}{3}\pi \bigg(3r^2 + 3r + 1\bigg) \\
    \end{align*}
\end{Answer}

\begin{Answer}
    \begin{align*}
        f(2+h) &= (2+h) - (2+h)^2 \\
               &= 2 + h - 4 + 4h + h^2 \\
               &= h^2 + 5h - 2
    \end{align*}
    \begin{align*}
        f(x+h) &= (x+h) - (x+h)^2 \\
               &= x + h - x^2 + 2hx + h^2 \\
               &= h^2 + h + 2hx - x^2 + x
    \end{align*}
    \begin{align*}
        \frac{f(x+h)-f(x)}{h} &= \frac{(h^2 + h + 2hx - x^2 + x) - (x - x^2)}{h} \\
        &= \frac{h^2 + h + 2hx}{h} \\
        &= h + 1 + 2x
    \end{align*}
\end{Answer}

\begin{Answer}
    \begin{align*}
        f(2+h) &= \frac{2+h}{2+h+1} \\
               &= \frac{2+h}{3+h}
    \end{align*}
    \begin{align*}
        f(x+h) &= \frac{x+h}{x+h+1}
    \end{align*}
    \begin{align*}
        \frac{f(x+h) - f(x)}{h} &= \frac{\frac{x+h}{x+h+1} - \frac{x}{x+1}}{h} \\
                                &= h^{-1} \left( \frac{(x+h)(x+1)}{(x+h+1)(x+1)} - \frac{x(x+h+1)}{(x+h+1)(x+1)}\right) \\
                                &= h^{-1} \left(\frac{x^2 + hx + x + h - x^2 - hx - x}{x^2 + hx + x + x + h + 1}\right) \\
                                &= h^{-1} \left(\frac{h}{x^2 + hx + 2x + h + 1}\right) \\
                                &= \frac{1}{x^2 + 2x + hx + h + 1}
    \end{align*}
\end{Answer}

\begin{Answer}
    The domain of $f$ is $\mathbb{R} - \{1/3\}$
\end{Answer}

\begin{Answer}
    The domain of $f$ is $\mathbb{R} - \{-2, -1\}$
\end{Answer}

\begin{Answer}
    The domain of $f$ is $[0, \infty)$
\end{Answer}

\begin{Answer}
    The domain of $g$ is $[0, 4]$
\end{Answer}

\begin{Answer}
    The domain of $h$ is all $x$ for which $x^2 - 5x > 0$, so either $x < 0$ or $x > 5$. The domain is $(-\infty, 0) \cup (5, \infty)$
\end{Answer}

\begin{Answer}
    The domain of this function is $[-2, 2]$, the range of this function is $[0, 2]$. I can't sketch it, but the graph is the upper half of a circle of radius $2$ centered at the origin.
\end{Answer}

\begin{Answer}
    Domain: $\mathbb{R}$
\end{Answer}

\begin{Answer}
    Domain: $\mathbb{R}$
\end{Answer}

\begin{Answer}
    Domain: $\mathbb{R}$
\end{Answer}

\begin{Answer}
    Domain: $\mathbb{R} - \{2\}$
\end{Answer}

\begin{Answer}
    Domain: $[5, \infty)$
\end{Answer}

\begin{Answer}
    Domain: $\mathbb{R}$
\end{Answer}

\begin{Answer}
    Domain: $\mathbb{R} - \{0\}$
\end{Answer}

\begin{Answer}
    Domain: $\mathbb{R} - \{0\}$
\end{Answer}

\begin{Answer}
    Domain: $\mathbb{R}$
\end{Answer}

\begin{Answer}
    Domain: $\mathbb{R}$
\end{Answer}

\begin{Answer}
    Domain: $\mathbb{R}$
\end{Answer}

\begin{Answer}
    Domain: $\mathbb{R}$
\end{Answer}

\begin{Answer}
    The slope is $(-6 -1) / 4 - (-2) = -7/6$. The domain is $[-2, 4]$. So 
    \[f(x) = -7/6(x+2) + 1 \qquad -2 \leq x \leq 4\]
\end{Answer}

\begin{Answer}
    \[f(x) = -5/9(x+3) + 1 \qquad -3 \leq x \leq 6\]
\end{Answer}

\begin{Answer}
    \[f(x) = -\sqrt{-x} + 1 \qquad x \leq 0\]
\end{Answer}

\begin{Answer}
    \[f(x) = +\sqrt{(1 - (x-1)^2}\]
\end{Answer}

\begin{Answer}
    \[f(x) = \begin{cases}
        x + 1 & -1 \leq x < 2 \\
        -\frac{3}{2}(x-2) & 2 \leq x \leq 4\\
    \end{cases}\]
\end{Answer}

\begin{Answer}
    \[f(x) = \begin{cases}
        -2x + 2 & 0 \leq x < 1 \\
        x - 1 & 1 \leq x \\
    \end{cases}\]
\end{Answer}

\begin{Answer}
    Let $l$ denote length and $w$ denote width. Then a perimeter of $20$ means
    \[20 = 2l + 2w\]
    The area of a rectangle is $A = lw$. We eliminate $w$ by solving for $w$ in the constraint above. We have that $w = (20 - 2l)/2 = 10 - l$. So our function is
    \[f(l) = l(10 - l)\]
\end{Answer}

\begin{Answer}
    Same as above, but solve for $w$ using the area constraint: $w = 16/l$. Then
    \[f(l) = 2l + 2\frac{16}{l} = 2l + \frac{32}{l}\]
\end{Answer}

\begin{Answer}
    The area of any triangle is the length of a side times the length of the altitude whose base is that side, divided by two. Let the length of a side be $s$. By symmetry, any altitude will bisect a side. Hence an altitude's height can be found by the Pythagorean theorem, with side lengths $s$ and $s/2$. The relationship between $s$ and $h$ is then
    \[ s^2 = h^2 + \left(\frac{s}{2}\right)^2\]
    So 
    \[h = \sqrt{s^2 - \frac{s^2}{4}} = \sqrt{\frac{3s^2}{4}} = \frac{s}{2}\sqrt{3}\]
    Then the area of the triangle is 
    \[f(s) = \frac{sh}{2} = \frac{s^2}{4}\sqrt{3}\]
\end{Answer}

\begin{Answer}
    The volume of a cube with edge length $s$ is $V = s^3$. The surface area is $6s^2$. We have that
    \[6s^2 = 6 \cdot (s^3)^{\frac{2}{3}}\]
    So 
    \[f(V) = 6V^{\frac{2}{3}}\]
\end{Answer}

\begin{Answer}
    Let $s$ be the length of a side of the base. Then its volume is $s^2 \cdot h$, where $h$ is the height of the box, so we have that
    \[2 = s^2h\]
    The surface of this box, since it is open topped, is $s^2 + 4sh$. We can write $h$ in terms of $s$ by solving the volume constraint for it. This gives us $h = 2/s^2$. Then
    \[f(s) = s^2 + 4s\left(\frac{2}{s^2}\right) = s^2 + \frac{8}{s}\]
\end{Answer}

\begin{Answer}
    The perimeter of the window is $30 = x + 2h + x\pi/2$, where $h$ is the height of the rectangular region. Then $h = 30 - x(1+\pi/2)$. The area of the window is $xh + \pi(x/2)^2$. Then
    \begin{align*} 
        f(x) &= x(30 - x(1+\pi/2)) + \pi\left(\frac{x}{2}\right)^2 \\
             &= 30x - \frac{2+\pi}{2}x^2 + \frac{\pi}{4}x^2 \\
             &= 30x + \frac{\pi - 2\pi - 4}{4} x^2 \\
             &= 30x - \frac{\pi + 4}{4}x^2
    \end{align*}
\end{Answer}

\begin{Answer}
    The volume of the box is $xlw$, where $l = 20 - 2x$ and $w = 12 - 2x$. Then our volume function is
    \[f(x) = x(20-2x)(12-2x)\]
\end{Answer}

\begin{Answer}
    \[f(x) = \begin{cases}
        2 & 0 < x \leq 1 \\
        2.2 & 1 < x \leq 1.1 \\
        2.4 & 1.1 < x \leq 1.2 \\
        2.6 & 1.2 < x \leq 1.3 \\
        2.8 & 1.3 < x \leq 1.4 \\
        3 & 1.4 < x \leq 1.5 \\
        3.2 & 1.5 < x \leq 1.6 \\
        3.4 & 1.6 < x \leq 1.7 \\
        3.6 & 1.7 < x \leq 1.8 \\
        3.8 & 1.8 < x \leq 1.9 \\
        4 & 1.9 < x < 2 \\
        \end{cases} = \begin{cases}
        2 & 0 < x \leq 1 \\
        2 + 0.2 \cdot\text{ceil}(10(x-1)) & 1 < x < 2
    \end{cases}\]
\end{Answer}

\begin{Answer}
    \begin{itemize}
        \item[(a)] 
            \[R(I) = \begin{cases}
                0 & I \leq 10000 \\
                \frac{I - 0.1\cdot(I-10000)}{I} & 10000 < I \leq 20000  \\
                \frac{I - 0.1\cdot(10000) - 0.15(I - 20000)}{I} & 20000 < I 
            \end{cases}\]
        \item[(b)]
            The amount of tax for $14000$ is $4000\cdot 0.1 = 400$ dollars. For $26000$ it's $10000\cdot 0.1 + 6000\cdot 0.15 = 1000 + 900 = 1900$ dollars.
        \item[(c)]
            \[T(I) = \begin{cases}
                0 & I \leq 10000 \\
                0.1(I-10000) & 10000 < I \leq 20000 \\
                1000 + 0.15(I-20000) & 20000 < I
            \end{cases}\]

    \end{itemize}

\end{Answer}

\begin{Answer}
    Skip
\end{Answer}

\begin{Answer}
    \begin{itemize}
        \item[(a)] 
            $g$ is even and $f$ is odd.
        \item[(a)] 
            $f$ is neither and $g$ is odd.
    \end{itemize}
\end{Answer}

\begin{Answer}
    The point $(-5, 3)$
\end{Answer}

\begin{Answer}
    The point $(-5, -3)$
\end{Answer}

\begin{Answer}
    \begin{itemize}
        \item[(a)] Mirror along $y$ axis
        \item[(b)] Mirror along $y$ axis and flip (the copy) along $x$ axis.
    \end{itemize}
\end{Answer}

\begin{Answer}
    Even. $f(-x) = (-x)^{-2} = ((-x)^2)^{-1} = (x^2)^{-1} = x^{-2}= f(x)$.
\end{Answer}

\begin{Answer}
    Odd. $f(-x) = (-x)^{-3} = ((-x)^3)^{-1} = (-(x^3))^{-1} = -x^{-3} = -f(x)$.
\end{Answer}

\begin{Answer}
    Neither. $f(1) = 2$, but $f(-1) = 0$.
\end{Answer}

\begin{Answer}
    Even.
\end{Answer}

\begin{Answer}
    Odd.
\end{Answer}

\begin{Answer}
    Neither.
\end{Answer}

\end{document}
